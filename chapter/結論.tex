%===================================================================================================
% まとめと今後の課題
%===================================================================================================
\chapter{結論}
本研究では,従来の環境地図データとRGB-Dカメラの計測データを用いてNDTによる屋内位置同定する方法を拡張し,環境地図とステレオカメラの計測データを用いてNDTによる屋内・屋外位置同定する方法について検討した.
本論文では,まずNDTとパーティクルフィルタ,地図データと計測データとのマッチング評価について述べた.その後,位置同定実験の際に使われるステレオカメラの設定,ステレオカメラを搭載した実験用台車,環境地図について述べた.その後,実際に実験用台車とステレオカメラを用いて実験を行い,提案手法の性能を評価した.\par
本研究の実験は大きく屋内実験と屋外実験に分けられる.屋内実験では,壁に人為的にテクスチャ画像を張らない通常の環境で実験を行ったテクスチャー無の実験と,ステレオカメラの特徴をより認識しやすくするために行ったテクスチャー有の実験を行った.テクスチャー有無の実験を通して,ステレオカメラを用いて最も位置同定精度が高かったパラメータを求め,各々の精度評価を行った.実験の結果,最も精度良くかつ高速であるパラメーター設定は地図・計測データオーバーラップ無であることが分かった.また,求めたパラメータを用いて代表的なRGB-DカメラであるKinect V2との精度比較を行った.これらの実験を通して,ステレオカメラを用いたNDTによる屋内位置同定が可能であることを確認した.

屋内実験の今後の課題として挙げられるのは,精度と計算速度の向上である.最も精度良くかつ高速であるパラメーター設定は地図・計測データオーバーラップ無であるが,ボクセル40cmの場合リサンプリング100回の位置同定に所要される時間はテクスチャー無の場合33秒,テクスチャー有の場合19秒で,収束性能はテクスチャー無の場合59\%で,テクスチャー有の場合は86\%であった.テクスチャー有の方が点群が多いが,計算時間が短い理由はテクスチャー無の場合,不確実的な点とノイズが多いため,無駄なところに点群のボクセルができてしまい,計算速度が遅くなると思われる.これについては,無駄の点群からボクセルを生成しないようにプログラムを組むことで改善できる.また,Kinect V2を用いた屋内位置同定がステレオカメラを用いた屋内位置同定より性能が良いことから,データのノイズ除去などのステレオカメラの性能向上も課題として挙げられる.

屋外実験では,まず屋外の日光の影響によるステレオカメラ・Kinect V2のデータ比較実験を行った.これらのデータから,屋外の日光の影響によりKinect V2はデータが撮れていないが,光のない夜になると,ステレオカメラのデータが撮れていないことが確認できた.しかし,夜になっても光さえあると,ステレオカメラのデータは撮れるため,屋外位置同定はステレオカメラを用いた方が有利であることが確認できる.次に,ステレオカメラ・Kinect V2のデータ比較実験を基にして実屋外環境でのステレオカメラを用いた位置同定実験を行った.その結果,屋外位置同定はオクルージョンの影響がない条件で,ランドマーク2以上と地面の特徴が撮れていると位置同定可能であることが分かった.最後に一本木と地面の特徴だけで位置同定可能有無を確かめる実験を行った.一本木と地面だけの特徴を用いて位置同定する場合には,,計測データを撮る方向によって精度が大きく違うことが分かった.

屋外実験の今後の課題として挙げられるのは,屋内実験の課題と同様に精度と計算速度の向上である.本研究で用いるステレオカメラは距離20mまでデータが撮れるが,ノイズを沢山含むため,10m以上のデータは消している.10m以上のデータを消さずに,ノイズを除去し位置同定を行えると,精度と計算速度の向上が期待される.また,屋外の地図データはパーティクルをランダムにばら撒くのには広すぎるため,今回の屋外実験では計測場所の半径3m以内にパーティクルをばら撒いている.今は初期値を知っているため,初期値の周りにパーティクルを集中させ位置同定が可能であるが,パーティクルを屋外地図データ全体にばら撒いてからの位置同定の収束性能を確かめる必要がある.