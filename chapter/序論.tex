%===================================================================================================
% 序論
%===================================================================================================
\chapter{序論} 
%---------------------------------------------------------------------------------------------------
\section{背景}
近年,少子化や高齢化のため,労働力不足の問題が深刻化している.この労働力不足の対策案として自律移動ロボットや無人搬送車,あるいは自動運転技術の開発が進められている.自律移動ロボットや自動運転技術が実現されることにより,労働力不足への対策のみならず,人の移動の効率化や交通事故の防止にも繋がると期待が寄せられている.自律移動ロボットや自動運転技術の実現には,高精度な位置同定と移動中の障害物回避技術が必要不可欠となる.そこで,本研究では移動ロボットの高精度な位置同定に注目する.
移動ロボットが自己位置推定を行う方法としては大きく環境地図が利用できない場合と利用できる場合に分けられる.環境地図が利用できない場合は,車輪の回転角度を積分し移動量を推定するオドメトリ,GPSなどの方法がある.環境地図が利用できる場合は外界センサ(レーザーやステレオカメラなど)の計測データと環境地図との比較をして自己位置推定を行う.

しかし,従来の自己位置推定方法にはいくつかの問題がある.オドメトリは車輪の滑り検出が難しかったり,誤差が起きると誤差が累積される問題がある.GPSは屋内など屋根があるところでは正確に自己位置推定ができない.環境地図が利用できる場合は累積誤差は小さいが,特に3次元位置同定の場合,3次元環境地図と外界センサの3次元計測データとの比較であるため,計算コストが莫大になる問題が生じる.そこでこれまでに,環境地図が利用できる場合の計算コストの問題をNormal Distribution Transform(NDT)を利用して解決した,RGB-Dカメラ(Kinect)を利用した高速な位置同定法[1]が提案されている.

\newpage
%---------------------------------------------------------------------------------------------------
\section{本研究の目的}

文献[1]では,代表的なRGB-DカメラであるKinectを用いて屋内の位置推定を行ってきたが,Kinectの性能上,屋外での位置推定は難しい.そこで,本研究では,屋外でも計測が可能なステレオカメラを用い,ステレオカメラから得られる3次元データを用いたNDTによる屋内・屋外の位置推定を目標とする.
%---------------------------------------------------------------------------------------------------
\section{概要}

本研究では,屋内のみならず,屋外環境でステレオカメラを用いた場合でも,従来提案されているNDTを用いた位置同定手法を適用することで,高速で高精度な移動ロボットの位置同定が可能であることを示す.このために以下の課題に対して検討を行う.

1つ目は,ステレオカメラを用いたNDTによる位置同定の可能性の検討である.前述の通り,ステレオカメラを用いたNDTによる位置同定は行われておらず,実際の屋内環境でどこまで位置同定できるかを検討する.
2つ目はステレオカメラの性能に関わるテクスチャー有無やボクセルの解像度,オーバーラップの有無などの各条件による精度と計算コストを比較し,最適な計測条件について検討する.ここで得られた最も効率的な条件を基にしてKinect V2との精度比較,屋外位置同定を行う.
最後は,屋外での適用可能性の検討である.移動ロボットに適用するためには屋内だけでなく屋外でも利用できなければならないため,ステレオカメラを用いた屋外位置同定の可能性について検討する.
%
\newpage
%---------------------------------------------------------------------------------------------------
\section{本論文の構成}
本論文はまず,NDTを用いた位置同定について述べてから,ステレオカメラを用いた位置同定について述べる.
次に計測実験と評価について述べた後,最後に本研究のまとめと今後の課題を述べる.

以下に本論文の構成を示す.
\begin{description}
 \item[第1章] 本研究の背景,目的,概要について述べる.
 \item[第2章] NDTを用いた位置同定について述べる.
 \item[第3章] ステレオカメラを用いた位置同定について述べる.
 \item[第4章] 計測実験と評価について述べる.
 \item[第5章] 本研究のまとめと今後の課題について述べる.
 \end{description}
