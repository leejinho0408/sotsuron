\documentstyle{yokou}
\type{卒業論文試問}
\title{ND Voxelを用いた\\ステレオカメラによる位置同定}
\author{李珍鎬}
\superviser{倉爪 亮 教授}
\date{平成 27年 2月 16日 (火) 10:30~10:45}
\place{システム情報科学府521号室}

\begin{document}
\maketitle

\if 0
近年,少子化や高齢化のため,労働力不足が深刻化されており,その解決策の一つとして人の移動をより効率的にする無人移動ロボットの開発が進められている.無人移動ロボットに必要不可欠となる技術は高精度な位置同定と障害物回避がある.その中でも,本研究では高精度な位置同定に注目して移動ロボットの位置同定について検討する,\par
移動ロボットの自己位置推定方法には大きく環境地図が利用できない場合と利用できる場合と二つに分けられる.利用できない場合の例としては,車輪の回転角度を積分するレーザーやカメラ画像から移動量を推定するオドメトリやGPSなどがある.環境地図が利用できる場合は外界センサ(レーザーやステレオカメラなど)の計測データと環境地図との比較をして自己位置推定を行う.しかし,従来の自己位置推定方法にはいくつかの問題がある.オドメトリは車輪の滑り検出が難しいかったり,誤差が起きると,誤差が累積されたりする問題をもつ.GPSは屋内など屋根があるところでは正確に自己位置推定ができない.また,環境地図が利用できる場合は累積誤差の問題は低いが,3次元環境地図をと外界センサの3次元計測データとの比較であるため,計算コストの問題が生じる.そこで,本研究ではNormal Distribution Transition(NDT)を用いて環境地図が利用できる場合の計算コスト問題を解決する.NDTは空間を格子状に分割し,それぞれの格子内に含まれる観測点群の位置に対して分散行列を求め,それを固有値分解することで得られる分散値及び軸の方向により点群を3次元正規分布で表現する手法である.NDTを用いることで,3次元データの簡略化ができる.\par
本研究はPanasonicとの共同研究であるため,Panasonicから提供されたステレオカメラを用いて計測データを撮る.また,地図データは高精度な3次元スキャナーであるFAROを用いて撮る.これまでは,Kinectを用いた屋内位置同定は行ってきたが,Kinectの性能上,屋外位置同定は難しい.そこで,本研究はステレオカメラの3次元データを用いてNDTによる屋内・屋外の位置同定を目標とする.
\fi
\if 0
近年,少子化や高齢化のため,労働力不足が深刻化されており,その解決策の一つとして人の移動をより効率的にする無人移動ロボットの開発が進められている.無人移動ロボットに必要不可欠となる技術は高精度な位置同定と障害物回避がある.その中でも,本研究では高精度な位置同定に注目して移動ロボットの位置同定について検討する,\par
移動ロボットの自己位置推定方法には大きく環境地図が利用できない場合と利用できる場合と二つに分けられる.利用できない場合の例としては,車輪の回転角度を積分するレーザーやカメラ画像から移動量を推定するオドメトリやGPSなどがある.環境地図が利用できる場合は外界センサ(レーザーやステレオカメラなど)の計測データと環境地図との比較をして自己位置推定を行う.しかし,従来の自己位置推定方法にはいくつかの問題がある.オドメトリは車輪の滑り検出が難しいかったり,誤差が起きると,誤差が累積されたりする問題をもつ.GPSは屋内など屋根があるところでは正確に自己位置推定ができない.また,環境地図が利用できる場合は累積誤差の問題は低いが,3次元環境地図をと外界センサの3次元計測データとの比較であるため,計算コストの問題が生じる.そこで,本研究ではNormal Distribution Transition(NDT)を用いて環境地図が利用できる場合の計算コスト問題を解決する.NDTは空間を格子状に分割し,それぞれの格子内に含まれる観測点群の位置に対して分散行列を求め,それを固有値分解することで得られる分散値及び軸の方向により点群を3次元正規分布で表現する手法である.NDTを用いることで,3次元データの簡略化ができる.\par
本研究はPanasonicとの共同研究であるため,Panasonicから提供されたステレオカメラを用いて計測データを撮る.また,地図データは高精度な3次元スキャナーであるFAROを用いて撮る.これまでは,Kinectを用いた屋内位置同定は行ってきたが,Kinectの性能上,屋外位置同定は難しい.そこで,本研究はステレオカメラの3次元データを用いてNDTによる屋内・屋外の位置同定を目標とする.
\fi
近年,少子化や高齢化のため,労働力不足が深刻化されており,その解決策の一つとして人の移動をより効率的にする無人移動ロボットの開発が進められている.無人移動ロボットに必要不可欠となる技術は高精度な位置同定と障害物回避がある.その中でも,本研究では高精度な位置同定に注目して移動ロボットの位置同定について検討する,\par
移動ロボットの自己位置推定方法には大きく環境地図が利用できない場合と利用できる場合と二つに分けられる.利用できない場合の例としては,車輪の回転角度を積分するレーザーやカメラ画像から移動量を推定するオドメトリやGPSなどがある.環境地図が利用できる場合は外界センサ(レーザーやステレオカメラなど)の計測データと環境地図との比較をして自己位置推定を行う.しかし,従来の自己位置推定方法にはいくつかの問題がある.オドメトリは車輪の滑り検出が難しいかったり,誤差が起きると,誤差が累積されたりする問題をもつ.GPSは屋内など屋根があるところでは正確に自己位置推定ができない.また,環境地図が利用できる場合は累積誤差の問題は低いが,3次元環境地図をと外界センサの3次元計測データとの比較であるため,計算コストの問題が生じる.そこで,本研究ではNormal Distribution Transition(NDT)を用いて環境地図が利用できる場合の計算コスト問題を解決する.NDTは空間を格子状に分割し,それぞれの格子内に含まれる観測点群の位置に対して分散行列を求め,それを固有値分解することで得られる分散値及び軸の方向により点群を3次元正規分布で表現する手法である.NDTを用いることで,3次元データの簡略化ができる.\par
本研究はPanasonicとの共同研究であるため,Panasonicから提供されたステレオカメラを用いて計測データを撮る.また,地図データは高精度な3次元スキャナーであるFAROを用いて撮る.これまでは,Kinectを用いた屋内位置同定は行ってきたが,Kinectの性能上,屋外位置同定は難しい.そこで,本研究はステレオカメラの3次元データを用いてNDTによる屋内・屋外の位置同定を目標とする.

\parindent=1zw

\end{document}
